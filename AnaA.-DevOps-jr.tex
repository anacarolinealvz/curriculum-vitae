\documentclass[11pt,a4paper]{article}

\usepackage[brazil]{babel}
\usepackage[utf8]{inputenc}
\usepackage[T1]{fontenc}
\usepackage[left=2cm, right=2cm, top=2cm, bottom=2cm]{geometry}
\usepackage{titlesec}
\usepackage{enumitem}
\usepackage{hyperref}
\usepackage{xcolor}
\usepackage{fontawesome5}

\definecolor{primary}{HTML}{1F2937}

\titleformat{\section}
  {\Large\bfseries\color{primary}}
  {}
  {0em}
  {}
  [\titlerule]

\pagestyle{empty}
\setlist[itemize]{noitemsep, topsep=2pt}

\begin{document}

% ---------- CABEÇALHO ----------
\begin{center}
    {\Huge \textbf{Ana Caroline Alves Assunção}}\\[6pt]
    {\large DevOps jr • Redes • Cloud}\\[8pt]

    \faMapMarker*~Belém - PA \quad
    \faPhone~(91) 98440-2755 \quad
    \faEnvelope~\href{mailto:anaassuncao942@gmail.com}{anaassuncao942@gmail.com}\\
    \faGithub~\href{https://github.com/anacarolinealvz}{github.com/anacarolinealvz}
    \quad
    \faLinkedin~LinkedIn: \href{https://www.linkedin.com/in/anacarolinealvz}{linkedin.com/in/anacarolinealvz}
\end{center}

% ---------- RESUMO ----------
\section*{Resumo Profissional}
Estudante de Engenharia de Telecomunicações e formada em Redes de Computadores,
com foco em \textbf{DevOps e Cloud Computing}.
Experiência prática com \textbf{monitoramento de sistemas (Zabbix)}, suporte de TI,
ambientes \textbf{Linux e Windows}, versionamento com \textbf{Git} e fundamentos de \textbf{AWS e Azure}.
Interesse em automação, infraestrutura, confiabilidade de sistemas e ambientes em nuvem,
buscando oportunidade como \textbf{DevOps Júnior ou Trainee}.

% ---------- EXPERIÊNCIA ----------
\section*{Experiência Profissional}

\textbf{Vibe Tecnologia} \hfill \textit{Nov/2025 – Atual}\\
\textit{Estagiária em Infraestrutura, Monitoramento e NOC}
\begin{itemize}
    \item Monitoramento de serviços, servidores e aplicações utilizando \textbf{Zabbix} com visualização de métricas via \textbf{Grafana}.
    \item Atuação em ambiente de \textbf{NOC}, realizando análise de alertas, identificação de incidentes e apoio na resolução de falhas.
    \item Criação de \textbf{automações com n8n} para melhoria de processos e rotinas do ambiente corporativo.
    \item Acesso e gerenciamento básico de \textbf{máquinas virtuais em ambiente de nuvem}, incluindo validações de serviços e recursos.
    \item Utilização da ferramenta \textbf{Control-M (BMC)} para acompanhamento e controle de execuções e fluxos automatizados.
\end{itemize}


\textbf{Polícia Civil – UFPA} \hfill \textit{Ago/2025 – nov/2025}\\
\textit{Estagiária em Desenvolvimento Back-end}
\begin{itemize}
    \item Desenvolvimento de funcionalidades em PHP para sistemas internos.
    \item Integração com PostgreSQL e otimização de consultas SQL.
    \item Criação e consumo de APIs REST.
    \item Versionamento de código com Git e GitHub/GitLab.
\end{itemize}

\textbf{Biblioteca Central – UFPA} \hfill \textit{Mar/2025 – Atual}\\
\textit{Estagiária em Suporte de TI}
\begin{itemize}
    \item Instalação e configuração de hardware e software.
    \item Suporte técnico remoto e presencial (RDP, TeamViewer, AnyDesk).
    \item Manutenção de computadores e infraestrutura.
    \item Noções de Active Directory (usuários e permissões).
\end{itemize}

\textbf{TV Liberal} \hfill \textit{Jan/2024 – Abr/2025}\\
\textit{Técnica de Sistemas em Broadcast}
\begin{itemize}
    \item Monitoramento de redes e serviços utilizando Zabbix.
    \item Manutenção de sistemas de transmissão e dados.
    \item Participação na implantação de novo centro de exibição.
\end{itemize}

% ---------- EDUCAÇÃO ----------
\section*{Formação Acadêmica}

\textbf{Engenharia de Telecomunicações} — UFPA \hfill \textit{2021 – 2026 (Previsão)}\\
Belém - PA

\textbf{Tecnologia em Redes de Computadores} — Wyden \hfill \textit{2023 – 2025}\\
Belém - PA

% ---------- CURSOS ----------
\section*{Cursos e Certificações}

\begin{itemize}
    \item \textbf{AWS Academy Cloud Foundations} — AWS Academy (2025)
    \item \textbf{Artificial Intelligence for Sustainable Projects – Towards COP 30} — 
    Institut d’Intelligence Artificielle Appliquée (160h, 2025)
    \item \textbf{DMA – Desk Manager Associate} — DLI Desk Manager (2025)
    \item \textbf{Configuração de Redes Complexas} — Wyden (240h)
    \item \textbf{Projetos de Redes de Computadores} — Wyden (240h)
    \item \textbf{Análise de Sistemas de TV} — Uniglobo (9h)
    \item \textbf{AWS Academy Cloud Foundations} — AWS (2025)
    \item \textbf{Configuração de Redes Complexas} — Wyden (240h)
    \item \textbf{Projetos de Redes de Computadores} — Wyden (240h)
    \item \textbf{Análise de Sistemas de TV} — Uniglobo (9h)
\end{itemize}

% ---------- HABILIDADES ----------
\section*{Habilidades Técnicas}

\textbf{DevOps \& Operação}\\
Monitoramento de sistemas com Zabbix e Grafana, atuação em NOC,
análise de alertas, troubleshooting básico e melhoria contínua de ambientes.

\vspace{4pt}
\textbf{Cloud Computing}\\
Ambientes em nuvem, acesso e validação de máquinas virtuais,
fundamentos de AWS, Azure e Google Cloud.

\vspace{4pt}
\textbf{Automação e Ferramentas}\\
Automações de rotinas com n8n, versionamento de código com Git e GitHub,
noções de pipelines e automação de processos.

\vspace{4pt}
\textbf{Sistemas Operacionais}\\
Linux e Windows, administração básica, serviços, permissões e logs.

\vspace{4pt}
\textbf{Redes e Infraestrutura}\\
Redes de computadores, TCP/IP, DNS, HTTP, cabeamento estruturado,
projetos de RF e fundamentos de infraestrutura.

\vspace{4pt}
\textbf{Banco de Dados}\\
Fundamentos de PostgreSQL.

\vspace{4pt}
\textbf{Ferramentas de Apoio}\\
Control-M (BMC), Power BI, Excel Avançado, VS Code, Trello.

\vspace{4pt}
\textbf{Infraestrutura como Código e Orquestração (em estudo)}\\
Ansible, Terraform e Kubernetes (AWS).

% ---------- IDIOMAS ----------
\section*{Idiomas}
Inglês — Intermediário

\end{document}
